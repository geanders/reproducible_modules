\documentclass[pdftex,english,12pt,parskip=half]{scrartcl}
\usepackage{palatino}
\usepackage{mathpazo}
\usepackage[margin=0.7in]{geometry}
%\usepackage{parskip}
\usepackage[compact]{titlesec}
\usepackage{amsmath,amssymb}
\usepackage{graphicx}
\usepackage{babel}
\usepackage{framed}
\usepackage{wrapfig}
\usepackage{subfig}
\usepackage[labelfont=bf,font=small,format=plain]{caption}
\usepackage{doi}
\usepackage{booktabs}
\usepackage{longtable}
\usepackage{multirow}
\usepackage[table]{xcolor}
\usepackage{wrapfig}
\usepackage{colortbl}
\usepackage{pdflscape}
\usepackage{tabu}
\usepackage{threeparttable}
\usepackage{threeparttablex}
\usepackage[normalem]{ulem}
\usepackage{makecell}
\usepackage{float}
%\usepackage[authoryear]{natbib}
\usepackage[numbers]{natbib}
\usepackage{url,hyperref,color}
\definecolor{darkblue}{rgb}{0.0,0.0,0.75}
\hypersetup{colorlinks,breaklinks,
            linkcolor=darkblue,urlcolor=darkblue,
            anchorcolor=darkblue,citecolor=darkblue}
\newcommand{\fixme}[1]{{\color{red} #1}}
\renewcommand\thesection{\Alph{section}}
\newcommand{\tabitem}{~~\llap{\textbullet}~~}
\begin{document}
\addtokomafont{section}{\large}
\def\bf{\normalfont\bfseries}
\pagestyle{empty}


{\large \textbf{Project Title:}}
%% 1. Project title – do not exceed 200 characters, including the spaces between words  and punctuation

Training modules for improving the reproducibility of experimental data recording and pre-processing with examples from microbiology and immunology

\clearpage


{\large \textbf{Cover Letter:}}

Please accept the attached grant ``Training modules for improving the reproducibility of experimental data recording and pre-processing with examples from microbiology and immunology'' by Drs. Brooke Anderson and collaborators in response to to \textit{RFA-GM-18-002: Training Modules to Enhance the Rigor and Reproducibility of Biomedical Research}.

\vspace{0.1in}

This submission is from Colorado State University. The proposal does not include Human Subjects, Vertebrate Animals, Biohazards, or Select Agents.

\vspace{0.1in}

This proposal [does what]. Therefore, it would be useful if one of the reviewers had [specific expertise]. 

\vspace{0.1in}

\noindent \textbf{Please assign the application to the following:}
\begin{description}
 \item \textbf{National Institute of General Medical Sciences (NIGMS).} 
   \textit{Rationale:} [Why]  
 \end{description}

\vspace{0.1in}

We thank you for the opportunity to submit this proposal. 

\vspace{0.2in}

Regards, 

\vspace{0.8in}

Brooke Anderson (PI) \\
Mike Lyons (Co-I) \\
Marcela (Co-I) \\
Mercedes (Co-I) \\
Others?

1681 Campus Delivery \\
Fort Collins, Colorado 80523-1681 \\
Telephone: 203-508-2738 \\
Email: brooke.anderson@colostate.edu \\

\clearpage


\textbf{Budget justification} \ \\
% Total direct costs over the three years of the project: $250,000
% May want to distribute as: $112,500 Year 01, $112,500 Year 02, $25,000 Year 03
% Modules should be published by the end of the second year. The third year 
% can be used for futher evaluation and refining of these modules.

{\large \textbf{Personnel}} \\

\noindent \textbf{Brooke Anderson,} \textit{Ph.D., Principal Investigator, x academic person-months, x summer person-months (20\% effort) in Years 01 through 02, x academic person-months, x summer person-months (5\% effort) in Year 03.} Dr. Anderson is an assistant professor of Epidemiology in the Department of Environmental \& Radiological Health Sciences at Colorado State University, with an affiliate position at the Department of Statistics. She is an expert in R programming and has created and published several open-source R packages, in particular to facilitate environmental epidemiological research. She has experience creating R programs to work with large data, including climate model output and large weather datasets, as well as programs that interface with open web-based datasets. She is the co-instructor of a series of Massive Open Online Courses on \textit{Mastering Software Development in R} through Coursera and an associated open online book. Dr. Anderson will lead the development of all training modules developed through this grant, including through supervising the development and integration of training materials from co-investigators. She will also lead user testing and other evaluation of all developed modules to ensure the developed modules are clear, effective, and well-matched to meet the needs of biological researchers from a variety of scientific backgrounds, including those new to programming. 

\noindent \textbf{Mike Lyons,} \textit{Ph.D., Co-Investigator, x academic person-months, x summer person-months (5\% effort) in Years 01 through 02, x academic person-months.}

\noindent \textbf{Marcela Henao-Tamayo,} \textit{Ph.D., Co-Investigator, x academic person-months, x summer person-months (5\% effort) in Years 01 through 02.}

\noindent \textbf{Mercedes Gonzalez-Juarrero,} \textit{Ph.D., Co-Investigator, x academic person-months, x summer person-months (5\% effort) in Years 01 through 02.}

{\large \textbf{Travel}} \\

\noindent \textbf{Domestic travel.} Funds for PI (Dr. Anderson) to travel to one domestic conferences (\$1,500 per conference) to learn about cutting edge reproducible research techniques. Potential conferences to be attended include the International R Users' Conference (UseR) on the years it is in the U.S. (typically every other year) or the Annual RStudio Conference. This domestic travel budget also includes funds (\$900 per trip) for Dr. Anderson (PI) to travel to up to two Program Meetings over the course of the project. No travel is anticipated in Year 3 of the project, when the focus will be on final evaluation and refining of the training developed and published in Years 01 and 02.
\begin{itemize}
\item Year 01: \$2,400
\item Year 02: \$2,400
\item Year 03: \$0
\end{itemize}
% The conference travel budget estimates $400 per flight, $175 per night at a hotel for four nights, and $75 per day for food for five days, then rounds up. 
% The SC meeting travel budget estimates $300 for flight, $175 per night at a hotel for 2.5 nights, and $75 per day for food for three days, then rounds down (presumably some meals would be covered).

{\large \textbf{Materials and Supplies}} \\ Annual funds (\$1,000 / year) are also  requested for recording equipment (e.g., microphone), screen-capture software (e.g., \textit{Camtasia}), other software, and books to facilitate the proposed training module development. \\

\noindent \textbf{Conference Registrations.} Funds are budgeted for Years 01 and 02 for Dr. Anderson to register for a yearly domestic conference on cutting-edge tools and principals for conducting computationally reproducible research (e.g., UseR or RStudio Conferences). Attending a conference each year will help us include the most up-to-date tools and approaches in the developed training modules.
\begin{itemize}
\item Year 01: \$500
\item Year 02: \$500
\item Year 03: \$0
\end{itemize}

\noindent \textbf{Hospitality.} Funds are budgeted each year to provide breakfast, lunch, coffee, and snacks for two days to 20 people (for the proposed annual extended user testing meetings at Colorado State University; budgeted at \$850 / year). Colorado State University provides room reservations free to faculty for similar events, and so funds to rent a space are not required (See Letter of Support, Dr. Jac Nickoloff).
\begin{itemize}
\item Year 01: \$850
\item Year 02: \$850
\item Year 03: \$850
\end{itemize}
% About $13 per "box" of coffee, serves about 12
% About $9 per person for catering from Spoons (for lunch)
% About $6 per person for catering from Panera for breakfast
% About $4 per person for snacks seems reasonable
% Breakfast for 20 x 2; Lunch for 20 x 2; Snacks for 20 x 2; Coffee for 20 x 4; 
% Total hospitality budget for extended user testing (round up): $800 
% Snacks for 15 x 3; Coffee for 15 x 3
% Total hospitality budget for each shorter user testing (round up): $75 

\noindent \textbf{Consulting?} For evaluation of the completed modules. Also, for recording demographics, etc., for evaluation purposes. Check the requirements in the grant call for this. \textit{Check STEM contact for this.}
% Maximum budget for evaluation costs: $3,000 over the project period

\newpage

{\large \textbf{Project Abstract:}}

We propose to develop training modules for improving the reproducibility of experimental data recording and pre-processing in scientific research. We aim to ensure these training modules are useful to laboratory-based researchers, who may have less prior training in open source software tools for reproducible research than researchers from biostatitics, epidemiology, and other disciplines. To ensure this, we will feature in these training modules examples from microbiology and immunology. We will create two sequences of modules that focus on improving computational reproducibility in the recording and pre-processing of experimental data, each containing approximately 12 modules, each featuring 5--30 minute videos, with supplemental online text, references, and practice exercises. The first sequence will be ``Improving the Reproducibility of Experimental Data Recording", and it will include modules on ... . The second sequence will be ``Improving the Reproducibility of Experimental Data Pre-Processing", and it will include modules on ... . These two sequences of training modules will be collectively published as an open online book using the \textit{bookdown} technology. Each module will form a chapter of this book, and will feature an embedded YouTube video of 5--30 minutes, with accompanying text in the book to provide trainees with a more detailed written reference they can refer to after completing the video module. Each module's chapter will conclude with practical exercises or open discussion questions to complement the material taught in the video. To ensure this material is completely free and open to researchers in the United States, we will publish this online book and the videos under a Creative Commons license.

\clearpage

{\large \textbf{Project Narrative:}}

We will develop training modules for improving the reproducibility of experimental data recording and pre-processing in scientific research. To ensure accessibility and relevance to laboratory-based researchers, we will feature in these training modules examples from microbiology and immunology. We will create approximately 25 short training modules collected in two sequences, one focusing on on reproducible approaches to recording experimental data and one on reproducible approaches to pre-processing experimental data. Each module will feature a video lecture of 5--30 minutes, and the full collection of video modules will be embedded in a free, open online book, with supplemental text and practical exercises to accompany each module. Training modules will be evaluated for researchers at a variety of training levels (undergraduates to faculty), drawing mainly from the Microbiology, Immunology, \& Pathology Department of Colorado State University.


\clearpage

\section*{Specific Aims}
\begingroup
    \fontsize{11pt}{12pt}\selectfont 

\textbf{Significance and educational aims of proposed modules.} Among the steps in conducting experimental research, the collection and pre-processing of experimental data are steps where it is critical to ensure good practices to ensure research reproducibility. However, much of the training available for computational reproducibility focuses on later steps of the research process, such as the analysis of processed experimental data. Here, we aim to develop training modules that focus on principals and techniques of improving the reproducibility of research at these stages of the research process. We will focus on making these modules accessible and useful to laboratory-based researchers by including examples of improving reproducibility at these stages of microbiology and immunology research projects and will evaluate the training modules through user testing among laboratory-based researchers.

\textbf{Proposed content of training modules.} We will develop two sequences of modules to train researchers in how and why to improve reproducibility in the collection and pre-processing of experimental data, including training on implementing these approaches using open source R software tools. Working with laboratory-based co-investigators on our team, we will ensure that these modules and the examples used in them are approachable and useful to researchers without extensive computational training, helping the modules address an audience beyond those of existing training materials on R programming tools for reproducible research. Further, the training modules will focus on reproducibility in the data collection and pre-processing stages of projects, rather than in experimental design or data analysis.

The first sequence will be ``Improving the Reproducibility of Experimental Data Recording", and it will include modules on the principals of the tidy data format, creating and using spreadsheet templates for data collection, developing project templates for organized and consistent data and meta-data collection, harnessing version control to improve transparency in data collection, and using GitLab for version-controlled collaborations. The second sequence will be ``Improving the Reproducibility of Experimental Data Pre-Processing", with modules on how and why to use code scripts for data pre-processing, creating reproducible data pre-processing protocols using \textit{rmarkdown}, examples of developing reproducible data pre-processing protocols, and converting from complex Bioconductor data types to tidy data formats to allow use of R's ``tidy" data tools. 

\textbf{Format of training modules.} Each module within each of these sequences will focus on a video lecture of 5--30 minutes. These videos will be collected together in an online book, where each module will form a chapter with an embedded video, and text and applied exercises or discussion questions will accompany each video. This book will be freely and openly published online under the Creative Commons license, to ensure it is available to any U.S. researcher. To ensure compliance with x, we will include in the online book transcripts for each training video.

\textbf{Evaluation of training modules.} We will conduct two-day user testing sessions each year of the grant. \textbf{These will be focused on scientists at a variety of levels (undergraduate to faculty) and will determine the usefulness, clarity, and relevance of the developed modules to these researchers.} These evaluation sessions will include live presentations of the lectures to be taped as modules, as well as directed work-throughs of the practical exercises included in the online book for each module. We will focus these testings on members of Colorado State University's Department of Microbiology, Immunology, \& Pathology. Following the user testing (e.g., six months--one year after), we will survey the lab head to determine which practices taught in the modules have been adapted.

\textbf{Project team.}  This project will bring together experts in R programming (Anderson, Lyons), including its use to improve the computational reproducibility of health-related research, with laboratory-based academic researchers in Microbiology and Immunology (Henao-Tamayo, Gonzalez-Juarrero) who are \textbf{attuned to the needs of and barriers to improving the reproducibility of experimental data collection and pre-processing}. Our team will allow us to develop training modules that both present state-of-the-art approaches and tools to reproducibility, but do so in a way that is prioritized to be most useful and accessible to health researchers whose training has focused on laboratory-related, rather than computational, methods, and for whom existing training materials on computational reproducibility might be hard to understand or apply to their own research projects. 

\endgroup

\clearpage

\section*{Research Education Program Plan}

\section{Significance}
\vspace{-0.1in}

\begin{quotation}
``Does the proposed program address a key audience and an important aspect or important need in training in rigor and reproducibility? Is there convincing evidence in the application that the proposed program will significantly advance the stated goal of the program?"
\end{quotation}

\section{Innovation}
\vspace{-0.1in}

\begin{quotation}
``Taking into consideration the nature of the proposed research education program, does the applicant make a strong case for this program effectively reaching an audience in need of the program's offerings? Where appropriate, is the proposed program developing or utilizing innovative approaches and latest best practices to improve the knowledge and/or skills of the intended audience?"
\end{quotation}

These modules will teach the principals of reproducibility as well as introduce researchers to tools for implementing reproducible research workflows. The implementation portion of these modules will focus on tools from the open-source R programming language. R can be freely, quickly, and easily downloaded and installed to a user's computer, allowing new users to get started quickly, a critical consideration for usable scientific software \cite{list2017ten}. R has been maintained for over a decade by the R Development Core Team and works with all major computing platforms, ensuring  widespread access, stability, and compatability, also critical for ease-of-use \cite{baumer2017lessons, altschul2013anatomy}. R offers a well-developed environment for creating new tools that extend the core language \cite{wickham2015r} and includes ample tools for documenting research workflows \cite{xie2015dynamic, xie2016bookdown}. R's status as the \textit{lingua franca} of statisticians and biostatisticians means that its use in early stages of experimental data recording and pre-processing can help foster closer collaborations between laboratory-based scientists and statisticians throughout the research process. R can be scaled as the volume of data in projects grows \cite{list2017ten}, as it includes tools to interface with distributed computing platforms (e.g., \textit{Hadoop} \cite{pathak2014rhadoop}, \textit{Spark} \cite{sparklyr}), and its scripts can be integrated within workflow management systems (e.g., \textit{Galaxy} \cite{goecks2010galaxy, walker2016models}). 

\section{Approach}

\begin{quotation}
``Does the proposed program clearly state its goals and objectives, including the audience to be reached, the content to be conveyed, and the intended outcome?  Is there evidence that the program is based on a sound rationale, as well as sound educational concepts and principles? Is the plan for evaluation sound and likely to provide information on the effectiveness of the program?"
\end{quotation}

\subsection{Proposed Research Education Program Plan}

\begin{quotation}
``While the proposed research education program may complement ongoing research training and education occurring at the applicant institution, the proposed educational experiences must be distinct from those research training and research education programs currently receiving federal support. When research training programs are on-going in the same department, the applicant organization should clearly distinguish between the activities in the proposed research education program and the research training supported by the training program. The research education proposed must be \textbf{targeted to trainees and investigators at any level}. State the \textbf{goals for education} and \textbf{justify the area of training} selected for module development in terms of its \textbf{relevance and potential impact} on improving the development of skills and knowledge important for conducting rigorous and reproducible research. Describe the \textbf{subject material} to be covered.  Describe the \textbf{format} for the training module proposed and \textbf{justify it in terms of the education goals}.  The \textbf{length} of the proposed training module should be explained in terms of \textbf{scope and depth of coverage} of the subject matter.  In addition, \textbf{how the research education will be utilized by trainees or investigators} should be described---for example, a module on how to avoid confirmation bias to be taken by all beginning laboratory workers, or a module on appropriate design of animal studies to be taken immediately prior to beginning such work.  Describe the \textbf{plans for piloting and evaluating the effectiveness} of the training module. Describe \textbf{plans for making the proposed training module section 508 compliant of the Rehabilitation Act} (29 U.S.C. '794 d), as amended by the Workforce Investment Act of 1998 (P.L. 105 – 220; see http://www.section508.gov/ for additional information). Provide a \textbf{timeline for module development, piloting and refinement, dissemination, evaluation, and maintenance}.  This timeline must propose \textbf{making the training publicly available within two years} of the award date."
\end{quotation}

\subsubsection{Educational goals of the modules}

The importance of computational reproducibility of scientific research is increasingly recognized by scientists, journals, and funding agencies, with such ``computationally reproducible" research requiring that all data and code for a research project be available and that this data and code can be used to regenerate study findings either by the original researcher or by other researchers \cite{ellis2017share, ram2013git}.

Every extra step of data formatting is another chance to introduce an error in the data. Therefore, by keeping research data pipelines simple---which can be more easily achieved if data is initially recorded in a format amenable to later data pre-processing, analysis, and visualization---researchers can decrease the potential for errors in the data and therefore improve the rigor and reproducibility of their research.

\textbf{Improving the Reproducibility of Experimental Data Recording} One key concept for improving the reproducibility of experimental data collection is understanding how to create and use the ``tidy" data format, which enables later data analysis using R's \textit{tidyverse} framework. The \textit{tidyverse} framework enables powerful and user-friendly data management, processing, and analysis by combining simple tools to solve complex, multi-step problems, and this framework is enabled by ensuring those simple tools share a common interface: a ``tidy" data format \cite{ross2017declutter, silge2016tidytext, wickham2016ggplot2, wickham2016r}. Working within the R framework facilitates research that adheres to standards of reproducibility through scriptable data analysis that can easily be placed under version control \cite{bryan2017excuse}. Since the tools are simple and share a common interface, they are easier to learn, use, and combine than tools created in the classical R framework \cite{ross2017declutter, lowndes2017our, reviewer2017review, mcnamara2016state}. This \textit{tidyverse} framework is quickly becoming the standard taught in introductory R courses and books \cite{hicks2017guide, baumer2015data, kaplan2017teaching, stander2017enthusing, reviewer2017review, mcnamara2016state} (see also Letters of Support [LOS], Kimmel, Peng), ensuring ample training resources for researchers new to programming, including books (e.g., \cite{baumer2017modern, lifesciencesR}, some freely available online, e.g., \cite{wickham2016r}), massive open online courses (MOOCs), onsite university courses \cite{baumer2015data, kaplan2017teaching, stander2017enthusing}, and Software Carpentry workshops \cite{wilson2014software, pawlik2017developing}. Further, tools that extend the tidyverse have been created to enable high-quality data analysis and visualization in several domains, including text mining \cite{silge2017text}, microbiome studies \cite{mcmurdie2013phyloseq}, natural language processing \cite{RJ-2017-035}, network analysis \cite{RJ-2017-023}, ecology \cite{hsieh2016inext}, and genomics \cite{yin2012ggbio}.

RStudio allows users to create their own custom ``Project" template, suited to a specific type of data analysis or software development, which can then be registered and accessed by other users \cite{rstudioprojecttemplate}. While a ``Project" can have any internal structure, a common structure can be enforced for a certain type of project through the creation of a new ``Project" template, which defines the required subdirectories, structure, and file names of common elements that must exist in the project \cite{rstudioprojecttemplate}. This template, when selected by a future user, will create a new directory with a ``skeleton" structure, potentially including templated files (e.g., for metadata). Projects saved in this format can be easily put under \textit{Git} version control in \textit{RStudio}, which includes a pane that allows users to work under version control without learning command-line version control language and, if desired, easily connect the project with an online version of the project hosted on \textit{GitHub}. This ``project" framework has recently been encouraged by a number of researchers as a way to enable computationally reproducible research, especially for research conducted by teams \cite{marwick2017packaging, parker2017opinionated, lowndes2017our}, and the use of \textit{Git} and \textit{GitHub} has also been encouraged as a tool to enable reproducible research \cite{piccolo2016tools, ram2013git, bryan2017excuse, lowndes2017our, cetinkaya2017infrastructure}.  

\textbf{Improving the Reproducibility of Experimental Data Pre-Processing.} Scriptable software tools bring key advantages compared to GUI software in terms of data pre-processing \cite{cetinkaya2017infrastructure, huber2015orchestrating, preeyanon2014reproducible, piccolo2016tools}, but it is critical to provide some training on the use of these tools for researchers new to programming. Expertise with a scripting language is not universal across the biomedical community, although literacy in programming is increasing in the sciences \cite{ram2013git}, and many now recommend programming as a critical skill for all biology Ph.D. students \cite{list2017ten}. 

\textit{Contrast the lack of guidance on experimental data recording for academic research with the guidelines from industry, including ``Good laboratory practice").}

\subsubsection{Module subject material}

We propose to develop two collections of modules, \textbf{Improving the Reproducibility of Experimental Data Recording} and \textbf{Improving the Reproducibility of Experimental Data Pre-Processing}. 


\begin{landscape}\begingroup\fontsize{10}{12}\selectfont
\rowcolors{2}{white}{gray!6}

\begin{longtable}[t]{>{\bfseries\raggedright\arraybackslash}p{10em}>{\raggedright\arraybackslash}p{28em}>{\raggedright\arraybackslash}p{14em}>{\raggedright\arraybackslash}p{3em}>{\raggedright\arraybackslash}p{14em}}
\caption{\label{tab:}\label{tab:content_one} Modules for the first sequence, \textbf{'Improving the Reproducibility of Experimental Data Recording'}. The color of each module's title indicates whether the module focuses on \textbf{Principles} (blue), \textbf{Implementation} (red), or \textbf{Case study examples} (black). This table is continued over several pages.}\\
\hiderowcolors
\toprule
Module title & Description of module content & Objectives (After taking the module, the trainee can ...) & Video length & Extra educational materials\\
\midrule
\endfirsthead
\caption[]{\label{tab:content_one} Modules for the first sequence, \textbf{'Improving the Reproducibility of Experimental Data Recording'}. The color of each module's title indicates whether the module focuses on \textbf{Principles} (blue), \textbf{Implementation} (red), or \textbf{Case study examples} (black). This table is continued over several pages. \textit{(continued)}}\\
\toprule
Module title & Description of module content & Objectives (After taking the module, the trainee can ...) & Video length & Extra educational materials\\
\midrule
\endhead
\
\endfoot
\bottomrule
\endlastfoot
\showrowcolors
\textcolor{blue}{\textbf{Separating data recording and analysis}} & Many biomedical laboratories currently use spreadsheets, with embedded macros, 
      to both record and analyze experimental data. This practice impedes the transparency
      and reproducibility of both data recording and data analysis. In this module, we 
      will describe this common practice and explain how it impedes the transparency and
      reproducibility of data recording and analysis. We will then outline alternative
      approaches that separate the steps of data recording and data analysis and explain
      how these alternative approaches can improve the reproducibility of biomedical 
      research. & \tabitem Explain the difference between data recording and data analysis 

     \tabitem Understand why collecting data on spreadsheets with embedded macros
        impedes transparency and reproducibility 

      \tabitem List alternative approaches that separate data recording and data analysis to 
        improve transparency and reproducibility & 15 & \tabitem Discussion questions about data recording approaches the trainee has 
      previously used in research projects and the benefits
      and limitations of those approaches in terms of data transparency and 
      reproducibility 

    \tabitem Short audio recording of two Co-Is giving their
      own answers to these discussion questions\\
\textcolor{blue}{\textbf{Principles and power of structured data formats}} & The format in which experimental data is recorded can have a large influence
      on how easy and likely it is to implement reproducibility tools in later stages of
      data pre-processing, analysis, and visualization. Recording data in a 'structured'
      format brings many benefits for later stages of the research process, 
      especially in terms of improving reproducibility.
      In this module, we will explain what makes a dataset 'structured' and
      why this format is a powerful tool for reproducible research. & \tabitem List the characteristics of a structured data format 

      \tabitem Describe how using a structured data format when recording experimental 
      data can improve the transparency and reproducibility of research

      \tabitem Outline other benefits of using a structured format when recording data & 10 & \tabitem Applied exercise: For example datasets, specify whether each is in a 
      structured data format and, in cases where it is not, draft a structured
      format that could be used to record the data 

    \tabitem Video walking trainees 
      through solutions to the applied exercise\\
\textcolor{red}{\textbf{The 'tidy' data format: an implementation of a structured data format}} & The 'tidy' data format is one implementation of a structured data format that
  was introduced in a 2014 paper and has since quickly 
  gained popularity among statisticians and data scientists. By consistently 
  using this data format, researchers have found they can employ combinations 
  of simple, generalizable tools to perform complex tasks in data processing, 
  analysis, and visualization. In this module, we will explain what characteristics determine
  if a dataset is 'tidy' and how use of the 'tidy' implementation of a structure 
  data format can improve the ease and efficiency
  of 'Team Science', including collaborations with statisticians. & \tabitem List characteristics defining the  
    the 'tidy' structured data format 

  \tabitem Explain the difference between the ideas of a structured data format (general 
    concept) and the 'tidy' data format (one implementation of that general format
    that is now particularly popular in data analysis) & 15 & \tabitem Quiz questions: For example datasets, correctly identify which of the 'tidy'
  data principles the dataset has or lacks 

  \tabitem Video giving answers and explanations
  for quiz questions, including showing 'tidy' versions of each example dataset 

  \tabitem Link to paper that established the 'tidy' data format\\
\textcolor{red}{\textbf{Designing templates for tidy data collection}} & This module will move from the principles of the 'tidy' data format to the 
      practical details of designing a 'tidy' data format to use when collecting 
      experimental data. We will describe common issues that prevent many real datasets from
      experimental research projects from being 'tidy' and show how these issues
      can be avoided when deciding the format in which to record experimental data.
      We will also provide rubrics and a checklist to help determine if a 
      data collection template complies with a 'tidy' format. & \tabitem Identify characteristics that keep a dataset from being 'tidy'
      
      \tabitem Convert data from an 'untidy' to a 'tidy' format & 20 & \tabitem Applied exercise: For an 'untidy' dataset, identify what 
      characteristics keep it from being 'tidy', and convert design a 'tidy' format

  \tabitem Video providing a detailed solution to the applied exercise\\
\textcolor{black}{\textbf{Example: Creating a template for 'tidy' data collection}} & In this module, we will walk through an example of creating a template to collect
      data in a 'tidy' format for a laboratory-based research project. As an example,
      we will use a research project headed by one of our Co-Is on drug efficacy in 
      murine tuberculosis models. We will walk through the 'untidy' format 
      initially used to collect data for this project, explain how this format 
      differed from a 'tidy' format, and show how we changed the format to be 'tidy'.
      Finally, we will show examples of how the experimental data can easily be 
      cleaned, analyzed, and visualized using reproducible tools once it is in a 
      'tidy' format. & \tabitem Understand how the principles of 'tidy' data can be applied 
      when recording experimental
      data for a real, complex research project;

      \tabitem List some advantages of using a 'tidy' data format for the example project & 15 & \tabitem Discussion questions, including listing examples of how experimental datasets
      the trainee has previously worked with or is currently working with are 'untidy' and
      how they could be converted to a 'tidy' format 

    \tabitem Short audio recording of two Co-Is giving their
      own answers to these discussion questions\\
\addlinespace
\textcolor{blue}{\textbf{Power of using a single structured 'Project' directory for storing and tracking research project files}} & To improve the computational reproducibility of a research project, researchers
      can use a single 'Project' directory to collectively store 
      all research data (raw and pre-processed), meta-data, code for data pre-processing,
      and research products further along the research pipeline (e.g., paper drafts, 
      figures, code for data analysis). In this 
      module, we will explain how using this practice from the 
      start of a research project improves the reproducibility of the projects, as well
      as facilitates other tools to improve reproducibility,
      including version control. Finally, we will 
      list some of the common components and subdirectories to include
      in the structure of a 'Project' directory, including subdirectories for raw and
      pre-processed experimental data. & \tabitem Describe a 'Project' directory, including common components and subdirectories 

      \tabitem List how collecting all research data and other files related 
      to a research project in a single 'Project' directory
      improves the reproducibility of a research project 

      \tabitem Describe how experimental data collection can be integrated with a
      research 'Project' directory & 20 & \tabitem Quiz questions: Test the trainee's understanding of a structured
      'Project' directory, what common components it may include, and the benefits
      of structuring research project files 
      within a single 'Project' directory from the beginning of the
      research project 

      \tabitem Video with detailed answers and discussion of quiz questions\\
\textcolor{red}{\textbf{Creating 'Project' templates}} & Researchers can use RStudio's 'Projects' interface to implement the structured
      collection of files for a research project in a single directory, with the added
      benefits that this interface facilitates use of version control. 
      Researchers can gain even more benefits, in terms of improving both the reproducibility
      and efficiency of research, by using a consistent structure for the 'Project' 
      directories for all of the research projects for a research group. We will demonstrate 
      how to implement structured project directories through RStudio,
      as well as how RStudio enables the creation of a template for all of a 
      research group's 'Project' directories, so a new project can be initialized
      with a skeleton directory that follows a directory format established
      by the research group. & \tabitem Be able to create a structured `Project` directory within RStudio 
      to use to consistently and reproducibly manage all files for a research project

     \tabitem Understand how RStudio can be used to create a template
      to use to create consistently-structured research 'Project' directories & 25 & \tabitem Discussion questions, including descriptions of how the trainee has saved and
      tracked research project files for previous research projects and what barriers,
      if any, these practices introduced in terms of the reproducibility and efficiency
      of research 

    \tabitem Short audio recording of two Co-Is discussing their answers to these questions\\
\textcolor{black}{\textbf{Example: Creating a 'Project' template}} & In this module, we will walk through a real example, based on the experiences of
      one of our Co-Is, of establishing the format 
      for a research group's 'Project' template, creating that template using RStudio,
      and initializing a new research project directory using the created template.
      This example will be from a laboratory-based research group that studies the efficacy of 
      tuberculosis drugs in a murine model. & \tabitem Create a 'Project' template in RStudio to use to initialize 
      consistently-formatted 'Project' directories to store all files related to 
      a research project
  
      \tabitem Initialize a new 'Project' directory using this template & 15 & \tabitem Applied exercise: Create and save a 'Project' 
      template that meets specifications provided for an example research group; 

     \tabitem Video demonstrating a detailed solution 
      to the applied exercise.\\
\textcolor{blue}{\textbf{Harnessing version control for transparent data recording}} & As a research project progresses, a typical practice in many experimental 
      research groups is to save new versions of files (e.g., 'draft1.doc', 'draft2.doc'),
      so that changes can be reverted. However, this practice 
      leads to an explosion of files, and it becomes hard to track 
      which files represent the 'current' state of a project. Version control allows
      researchers to edit and change research project files more cleanly, while maintaining
      the power to 'backtrack' to previous versions. Further, with version control,
      messages can be included to explain any changes.
      In this module, we will explain what version
      control is and how it can be used in research projects to improve the transparency 
      and reproducibility of research, particularly for transparent data recording. & \tabitem Describe version control and what it does 

      \tabitem Explain how version control can be used to improve reproducibility at 
      the data recording stage of research & 10 & \tabitem Discussion questions, including discussion of how the trainee has 
      managed evolving research project files in previous projects and any barriers
      those practices introduced in conducting efficient and reproducible research 

      \tabitem Short audio recording of two Co-Is giving their
      own answers to these discussion questions\\
\textcolor{blue}{\textbf{Enhance the reproducibility of collaborative research with version control platforms}} & Once a researcher has learned to use git on their own 
      computer for local version control, they can begin using version control 
      platforms (e.g., GitLab, GitHub) to collaborate with others in their research
      group while keeping the project under version control. These platforms allow
      the all collaborators to share a current version of a project directory 
      (similar to Dropbox), but in a way that allows easy use of version control 
      and that is more efficient for exploring (and, when necessary, undoing) the changes 
      each team member has made to project files. In this module, we will describe 
      why a research team may want to use a version control platform like GitLab 
      to work collaboratively on a project. & \tabitem List the benefits of using a version control platform like GitLab, rather 
      than Dropbox, to share project files, 
      particularly in terms of improving transparency and reproducibility 

     \tabitem Describe the difference between version control (e.g., git) and 
      a version control platform (e.g., GitLab) & 10 & \tabitem Discussion questions: Describe how you have shared research project 
    files in past research projects---email? Dropbox? Department servers?

    \tabitem Short audio file with two Co-Is discussing their answers\\
\textcolor{red}{\textbf{Using git and GitLab to implement version control}} & For many years, use of version control required use of the command line,
  limiting its accessibility to researchers with limited programming experience.
  However, graphical interfaces have removed this barrier, and RStudio has 
  particularly user-friendly tools for implementing version control.
  In this module, we will show how to use 
  \textit{git} through RStudio's user-friendly interface and how to connect from a local
  computer to \textit{GitLab} through RStudio. & \tabitem Understand how to set up and use \textit{git} through RStudio's interface 

  \tabitem Understand how to connect with \textit{GitLab} through RStudio to collaborate on  
  research projects while maintaining version control & 20 & \tabitem Applied exercise: Use RStudio to 
  initialize \textit{git} version control for a directory 
  and to make several tracked changes. Create a matching \textit{GitLab} repository and use
  RStudio to push local changes to this GitLab version of the directory

  \tabitem Video 
  walking trainees through a detailed solution to the exercise\\*
\end{longtable}
\rowcolors{2}{white}{white}\endgroup{}
\end{landscape}



\begin{landscape}\begingroup\fontsize{9}{11}\selectfont
\rowcolors{2}{white}{gray!6}

\begin{longtable}[t]{>{\bfseries\raggedright\arraybackslash}p{10em}>{\raggedright\arraybackslash}p{30em}>{\raggedright\arraybackslash}p{15em}>{\raggedright\arraybackslash}p{3em}>{\raggedright\arraybackslash}p{15em}}
\caption{\label{tab:}Modules for \textbf{'Improving the Reproducibility of Experimental Data Pre-Processing'}. The color of each module's title indicates whether the module focuses on \textbf{Principals} (blue), \textbf{Implementation} (red), or \textbf{Case study examples} (black).}\\
\hiderowcolors
\toprule
Module title & Description of module content & Objectives (After taking the module, the trainee can ...) & Video Length & Extra educational materials\\
\midrule
\endfirsthead
\caption[]{Modules for \textbf{'Improving the Reproducibility of Experimental Data Pre-Processing'}. The color of each module's title indicates whether the module focuses on \textbf{Principals} (blue), \textbf{Implementation} (red), or \textbf{Case study examples} (black). \textit{(continued)}}\\
\toprule
Module title & Description of module content & Objectives (After taking the module, the trainee can ...) & Video Length & Extra educational materials\\
\midrule
\endhead
\
\endfoot
\bottomrule
\endlastfoot
\showrowcolors
\textcolor{blue}{\textbf{Principals and benefits of scripted pre-processing of experimental data}} & The experimental data collected for biomedical research often requires 
      pre-processing before it can be analyzed (e.g., gating of flow cytometry data, 
      peak finding and quantification for LC / MS metabolomics data). While 
      often proprietary, point-and-click software is available for this pre-processing,
      use of such software can limit the transparency and reproducibility of this 
      pre-processing stage of the analysis, and point-and-click software is often 
      time-consuming to use for repeated tasks over large research projects.
      In this module, we will explain how using scripts to apply open source software 
      for this pre-processing step can improve the transparency, reproducibility, and
      transparency of research. & \tabitem Define pre-processing of experimental data and give some examples; 

      \tabitem Describe how the use of proprietary software for pre-processing experimental
      data limits transparency and reproducibility; 

      \tabitem Understand what an open source
      code script is and how it can be used as an alternative in pre-processing 
      experimental data; 

      \tabitem List some popular packages in R that can be used to 
      pre-process biomedical experimental data. & 15 & \tabitem Discussion questions, including discussion of which steps are commonly used to 
      pre-process experimental data in the trainee's research area; 
      
      \tabitem Short audio recording of two Co-Is giving their
      own answers to these discussion questions; 
      
      \tabitem List of some popular R packages for
      pre-processing different types of biomedical experimental data.\\
\textcolor{red}{\textbf{Introduction to R code scripts}} & In this module, we will explain the difference between interactive software use and the
      use of code scripts, using examples from R. We will then demonstrate how to 
      create, save, and run an R code script for a simple data cleaning task. & \tabitem Describe what an R code script is and how it differs from interactive
      coding in R; 

      \tabitem Create and save an R script to perform a simple data 
      pre-processing task; 
  
      \tabitem Run an R script. & 10 & \tabitem Applied exercise: Given a simple example dataset and a data cleaning task, 
      write and run an R script to perform the task. Then adapt that script to re-use
      it on a second, similar example dataset. Hints on useful R functions will be 
      provided to help trainees new to the R language; 

      \tabitem Video providing a detailed
      walk-through of a solution to the applied exercise.\\
\textcolor{red}{\textbf{Simplify scripted pre-processing through R's 'tidyverse' tools}} & The R programming language now includes a collection of 'tidyverse' extension 
      packages that enable user-friendly yet powerful work with experimental data,
      including pre-processing and exploratory visualizations. The principal behind
      the 'tidyverse' is that a collection of simple, general tools can be joined 
      together to solve complex problems, as long as a consistent format is used 
      for the input and output of each tool (the 'tidy' data format taught in other
      modules). In this module, we will explain why this 'tidyverse' system is so
      powerful and how it can be leveraged within biomedical research, especially for
      reproducibly pre-processing experimental data. & \tabitem Define R's 'tidyverse' system; 

      \tabitem Explain how the 'tidyverse' collection
      of packages can be both user-friendly and powerful in solving many complex
      challenges in working with data; 

      \tabitem Describe the difference between 'base R' and
      R's 'tidyverse'. & 15 & \tabitem Quiz: Questions will test the trainee's understanding of what R's 
      'tidyverse' is and why it is a powerful yet user-friendly tool for improving
      the reproducibility, transparency, and efficiency of research projects. 

      \tabitem Video with detailed answers and explanations for the quiz questions; 

      \tabitem Links to further free sources for developing more 'tidyverse' coding 
        skills.\\
\textcolor{blue}{\textbf{Complex data types in experimental data pre-processing}} & Raw data from many biomedical experiments, especially those that
  use high-throughput techniques, can be very large and complex. Because of the 
  scale and complexity of these data, software for pre-processing the data in R
  often uses complex, 'untidy' data formats. These complex data formats are necessary
  for computational efficiency and to aid the structure of the pre-processing
  software, but the 'untidy' formats add a critical barrier for researchers who 
  wish to explore and visualize the data. In this module, we will 
  describe the complex data formats are often used in open source software for 
  pre-processing experimental data, explain why use of these complex formats is
  often necessary, and outline how these complex formats create hurdles in 
  implementing reproducibility tools among laboratory-based scientists. & \tabitem Explain why R software for pre-processing biomedical data often stores the 
  data in complex, 'untidy' formats; 
  
  \tabitem Describe how these complex data formats can create barriers to 
  laboratory-based researchers seeking to use reproducibility tools for 
  data pre-processing. & 15 & \tabitem Quiz: Determine trainee's understanding of why complex data formats
  are often used within steps of experimental data pre-processing in open-source
  software; 
  
  \tabitem Video providing detailed
  answers to quiz questions.\\
\textcolor{red}{\textbf{Complex data types in R and Bioconductor}} & Many R extension packages for pre-processing experimental data use complex (rather than
    'tidy') data formats within their code, and many output data in complex formats. This
    is necessary for computational efficiency of the pre-processing, but creates a hurdle
    for using many common tools taught to improve research reproduciblity, 
    including R's 'tidyverse' tools. With the rising popularity of the 'tidyverse' collection of R tools, which require
      data to be in a 'tidy' format, R users have recognized that the use of complex, 'untidy'
  data formats can complicate reproducible code for data pre-processing, analysis,
  and visualization. Very recently, some researchers have developed tools 
  (the broom and biobroom R package extensions) that
  can extract a 'tidy' dataset from data stored in a complex, list-based format.
  These tools create a clean, simple connection between the complex data formats
  often used in pre-processing or modeling experimental data and the 'tidy' format
  required to use the 'tidyverse' tools now taught in many introductory R courses. In this module, we will describe the 'list' data structure,
    the common backbone for complex data structures in R, and well as provide tips on how to
  explore and extract data stored in R in this format. 
      We will then demonstrate how the new \textit{broom} and \textit{biobroom} packages 
    can be used to extract  to use  to convert output from pre-processing software to 'tidy'
    data formats for futher steps of reproducible data visualization and analysis. 
      'tidy' versions of pre-processed experimental data from their complex data formats,
      to allow user-friendly data analysis and visualization using the widely-taught
      general 'tidyverse' tools. & \tabitem Describe the structure of R's 'list' data
      format; 

      \tabitem Take basic steps to explore
      and extract data stored in R's complex, list-based structures;
  
      \tabitem Describe what the \textit{broom} and \textit{biobroom} R packages can do; 

      \tabitem Explain why 
  converting data from a complex format to a 'tidy' format can improve the 
  transparency and reproducibility of a research project. & 15 & \tabitem Applied exercise: We will provide example data in a complex, list-based format. 
  The trainee will explore this data based on step-by-step instructions and will 
  extract specified elements from the data format as well as practice using \textit{broom} and
  \textit{biobroom} R packages to extract 'tidy' data from complex data formats.; 
  
  \tabitem Video providing a detailed
  walk-through of completing this exercise, with explanations for specific steps.\\
\addlinespace
\textcolor{black}{\textbf{Example: Converting from complex data types to 'tidy' data formats}} & We will provide a detailed example of a case where data pre-processing in R
      has resulted in data in a complex, 'untidy' format, and where tools can be 
      used to extract data in a 'tidy' format, which then can easily integrate
      with general R 'tidyverse' tools for data analysis and visualization. We will
      walk through an example of applying automated gating to flow cytometry data. 
      We will demonstrate the complex initial format of this pre-processed data and then
      show trainees how a 'tidy' dataset can be extracted and used for further data
      analysis and visualization. This example will use real experimental data from 
      research on the immunology of tuberculosis [more details on this project]. & \tabitem List R package extenstions that can be used to extract 'tidy' data from 
      complex, 'untidy' R data formats; 

      \tabitem Describe how these tools can be used in 
      research projects to shift from data pre-processing to analysis and visualization
      of the processed data. & 20 & \tabitem Applied exercise: Trainees will be given an example dataset in a complex, 
      'untidy' data format in R and will be instructed in how to convert it to 
      a 'tidy' format and then create some straightforward plots of the data based on 
      this 'tidy' dataset; 

      \tabitem Video demonstrating a detailed solution to the applied
      exercise.\\
\textcolor{blue}{\textbf{Introduction to reproducible data pre-processing protocols}} & Reproducibility tools can be used to create reproducible data pre-precessing 
    protocols---documents that combine code and text in a 
  'knitted' document ... . In this module, we will describe how
  reproducible data pre-processing protocols 
  can be leveraged early in a research project to improve the reproducibility 
  of the pre-processing of experimental data and to ensure transparency, consistency,
  and reproducibility across the research projects conducted by a research team. & \tabitem Describe a reproducible data pre-processing protocol; 
  
  \tabitem Explain how reproducible data pre-processing protocol can be used to improve
    the reproducibility
  of research projects at the data pre-processing phase; 
  
  \tabitem List other benefits of using reproducible data pre-processing protocols,
    including improving efficiency and consistency of data pre-processing across a
    research groups research projects. & 15 & \tabitem Discussion questions: Including discussion of how reproducible data pre-processing 
  protocols can make biomedical research more reproducible at the data pre-processing stage; 
  
  \tabitem Short audio 
  recording of two Co-Is giving their
  own answers to these discussion questions.\\
\textcolor{red}{\textbf{Introduction to RMarkdown as a tool for creating reproducible data pre-processing protocols}} & RMarkdown can be used to create documents that combine code and text in a 
      'knitted' document, and it has become a popular tool among statisticians
      and data scientists for improving the computational reproducibility and 
      efficiency of their research. This tool can also be used earlier in the 
      research process, however, to develop well-documented code to pre-process
      raw experimental data. In this module, we will show trainees the types of 
      documents that can be created and run using RMarkdown. We will describe how
      RMarkdown is used among statisticians to improve the reproducibility, 
      efficiency, and transparency of data analysis, as well as describe how it 
      can be leveraged earlier in a research project to improve the reproducibility 
      of the pre-processing of experimental data. We will also provide detailed instructions on how to use RMarkdown
      in RStudio to create documents that combine code and text. We will explain how
  these documents can be converted into different final file formats (PDF, HTML,
  Microsoft Word). We will show how an RMarkdown document describing a data 
  pre-processing protocol can be used to efficiently apply the same data
  pre-processing steps to different sets of raw data. & \tabitem Define RMarkdown; 

      \tabitem Describe the documents that can be created using
      RMarkdown; 

      \tabitem Explain how RMarkdown can be used to improve the reproducibility
      of research projects at the data pre-processing phase; 
  
      \tabitem Create a document in RStudio using 
      RMarkdown; 
  
  \tabitem Render the document in 
  multiple file formats; 
  
  \tabitem Apply the document to several different datasets
  that follow the same format. & 15 & \	abitem Applied exercise: Trainees will be asked to create, save, and render 
    their own RMarkdown document through RStudio; 
  
  \tabitem Video providing a detailed
  walk-through of a solution to the applied exercise.\\
\textcolor{black}{\textbf{Example: Creating a reproducible data pre-processing protocol}} & We will provide an example of creatin a reproducible protocol for the automated
      gating of flow cytometry data for a project on the immunology of tuberculosis
      [more details on project]. This data pre-processing protocol was created 
      using RMarkdown and allows the efficient, transparent, and reproducible 
      gating of flow cytometry data for all experiments in a research project. We will
      walk the trainees through the final pre-processing protocol, how we apply this
      protocol to new experimental data, and how we developed the protocol initially. & \tabitem Explain how a reproducible data pre-processing protocol can be integrated
      into a real research project; 

      \tabitem Describe what is included in a data 
      pre-processing protocol; 

      \tabitem Understand how to design and implement a data
      pre-processing protocol to replace manual or point-and-click data pre-processing
      tools. & 20 & \tabitem Quiz questions: These will test the trainees understanding of how and why we 
      created well-documented and reproducible data pre-processing protocols for this 
      project, as well as how this helps increase the transparency and reproducibility
      of the research project; 

      \tabitem Short audio recording of discussion with the head of
      this example research project on how this reproducible data pre-processing fits into
      her research project and how use of this protocol differs from previous data
      pre-processing practices in the group.\\*
\end{longtable}
\rowcolors{2}{white}{white}\endgroup{}
\end{landscape}


The \textbf{Improving the Reproducibility of Experimental Data Pre-Processing} collection will include the following modules:

\begin{enumerate}
\item An introduction to R code scripts
\item The benefits of scripts for data pre-processing
\item Getting started with RMarkdown 
\item The relationship between R code scripts and RMarkdown documents
\item Creating reproducible data pre-processing protocols using Rmarkdown
\item Example of a reproducible data pre-processing protocol: Automated gating for flow cytometry data
\item Example of a reproducible data pre-processing protocol: Measuring metabolite feature intensities for metabolomics LC/MS data
\item Complex data types in R and their use in Bioconductor packages
\item Converting from complex data types to ``tidy" formats for data analysis and visualization with R's ``tidy" data tools
\end{enumerate}

\subsubsection{Format for the training modules}

\begin{itemize}
\item Online book created through the ``bookdown" format, with each module as a book chapter. We can use Git Pages to host this (CSU options for web hosting?).
\item Training videos embedded for each module, each 5--30 minutes. Videos will be similar to online course lectures and will be hosted using YouTube. Embedding in the book will allow users to watch videos without leaving the book's webpage. 
\item Each chapter will end with exercise questions (around 10 questions, combination of discussion questions and applied exercises), as well as an embedded video with discussion of the discussion questions and a detailed walk-through of answers to applied exercises. 
\item Possibly host this through an online course platform like DataCamp?
\end{itemize}

\textbf{Online book.} To ensure that these training modules are easy for researchers to access, use, and reference, we will provide all training materials through an online book created with the \textit{bookdown} framework \cite{xie2016bookdown} (see LOS, Xie). Through this new framework, we will be able to create a searchable online book that weaves R code into the text and that can include embedded tutorial videos, active weblinks to online references, and computationally reproducible practice examples and exercises. Further, by including R code examples as executable code, we will be able to use this online book to frequently check tutorial code examples to quickly identify and fix any broken tutorial code \cite{xie2016bookdown}.  Dr. Anderson (PI) has previously created two \textit{bookdown}-based books, \textit{R Programming for Research} and \textit{Mastering Software Development in R}.  

\subsubsection{Piloting and evaluating effectiveness of training modules}

\subsubsection{Insuring compliance with Rehabilitation Act}

\subsection{Team}

\subsubsection{Program Director/Principal Investigator}

\begin{quotation}
``Is the PD/PI capable of providing both administrative and scientific leadership to the development and implementation of the proposed program? Is there evidence that an appropriate level of effort will be devoted by the program leadership to ensure the program's intended goal is accomplished? If the project is collaborative or multi-PD/PI, do the investigators have complementary and integrated expertise; are their leadership approach, governance and organizational structure appropriate for the project?"
\end{quotation}

\begin{quotation}
``Describe \textbf{arrangements for administration} of the program.  Provide evidence that the Program Director/Principal Investigator is actively engaged in research and/or teaching in an area related to the mission of NIH, and can \textbf{organize, administer, monitor, and evaluate the research education program}. For programs proposing multiple PDs/PIs, describe the complementary and integrated expertise of the PDs/PIs; their leadership approach, and governance appropriate for the planned project."
\end{quotation}

\subsubsection{Other members of the team}

\subsection{Institutional Environment and Commitment}

\begin{quotation}
``Describe the institutional environment, reiterating the \textbf{availability of facilities and educational resources} (described separately under Facilities \& Other Resources), that can contribute to the planned Research Education Program. Evidence of institutional commitment to the research educational program is required. A \textbf{letter of institutional commitment} must be attached as part of Letters of Support (see below). Appropriate institutional commitment should include the provision of adequate staff, facilities, and educational resources that can contribute to the planned research education program."
\end{quotation}

\subsection{Evaluation Plan}

\begin{quotation}
``Applications must include a plan for evaluating the activities supported by the award in terms of their \textbf{frequency of use} and their \textbf{usefulness}. The use of \textbf{multiple evaluation approaches} is highly encouraged as is \textbf{testing several groups with different characteristics}. The application must specify \textbf{baseline metrics (e.g., numbers, educational levels, and demographic characteristics of test group)} in a structured format, as well as \textbf{measures to gauge the short and long-term success of the research education award in achieving its objectives}. Applicants are expected to \textbf{obtain feedback from test group} to help identify weaknesses and to provide suggestions for improvements, and \textbf{make the evaluation and feedback data} available to NIGMS staff."
\end{quotation}

\textbf{Learning objectives} These are what we're trying to determine were achieved by the training modules.

\textbf{Pilot / text group evaluation}:

\begin{itemize}
\item Work with GAUSSI to use some students as pilot testers?
\item Recruit researchers / faculty as pilot testers?
\item Work with CSU's Research Ethics group to figure out ways to pilot?
\end{itemize}

The key goal of this project is to develop tools that are easy to use by a broad range of applied metabolomics researchers. We will therefore conduct regular short (two hours) user testings several times per year and one long (two days) user testing session per year. The user testing groups will consist primarily of student trainees involved in applied metabolomics research from a range of departments at Colorado State University (see LOS, Clark, De Long, Heuberger). The shorter testing sessions will be used to test stable versions of the developed R packages immediately before they are published to the Comprehensive R Archive Network (CRAN). These testing sessions will ask participants to work through package tutorial vignettes and other test cases and will focus on identifying aspects of the package that cause unwanted behavior on certain computer systems or when users provide unexpected input. Further, these shorter testing sessions will be used to identify sections of vignette tutorials or help files that are unclear to targeted users. The longer, two-day testing sessions will be more open and will provide participants with open-ended metabolomics data analysis challenges, using data from \textit{Metabolomics Workbench} and the \textit{National Metabolomics Data Repository} that differ from the data used to develop the tools. Participants will be scheduled into groups, allowing them to participate during the two days while meeting outside obligations like classes and meetings. Participants will be encouraged to use both the tools we develop here, as well as any other available R tools, to complete these challenges in groups in a ``hackathon"-style structure. Participants will be given guidance on how to use GitHub to work in groups and share final results from the challenge, a framework Dr. Anderson has successfully used in a previous similar event at Colorado State University (Figure \ref{csu-r-hackathon}). This longer testing will help us identify limitations in the usability of the tools we will develop here, validate the tools using separate data, scope future development aims by identifying analysis tools that users would have liked to have to help with these open-ended challenges but that were not yet available in the R environment, and compare the tools we develop to existing metabolomics data analysis and visualization tools. Dr. Anderson (PI) has run several two-hour user testing sessions with students from various departments of Colorado State University prior to releasing R software packages \cite{futureheatwaves, countyweather}. Further, in April 2016, she led a longer, two-day user testing session through a Weather Data Hackathon at Colorado State University (Figure \ref{csu-r-hackathon}). Around 15 people participated, including undergraduate students, graduate students, postdoctoral fellows, and professors from CSU's Departments of Atmospheric Sciences, Civil \& Environmental Engineering, Microbiology, and Statistics. Some of the ideas and code developed during this Hackathon have since led to development and publication of open source software \cite{countyfloods, noaastormevents}.

\begin{figure}[h]
\begin{center}
\includegraphics[width=0.5\textwidth]{figures/csu_hackathon.png}
\end{center}
\caption{\label{csu-r-hackathon} Some of the approximately 15 undergraduate students, graduate students, postdoctoral fellows, and professors who participated in a two-day Weather Data Hackathon at Colorado State University in April 2016.}
\end{figure}

\textbf{Long-term evaluation}:

\begin{itemize}
\item Google Analytics for online book. How often are people accessing the book? How long are they spending on the book website? Where are the people accessing the book?
\item YouTube analytics for the embedded videos. How often are people accessing the book? How long are they spending on the book website? Where are the people accessing the book?
\item Quiz for each chapter of the book? Use to evaluate how well they've mastered the material? (Possibly could use embedded Shiny apps for this? Other ways to do this?)
\item Rating options for each chapter of the online book? Usefulness? What they learned?
\item Survey within each chapter of the online book? Educational level, demographic characteristics.
\end{itemize}

\subsection{Dissemination Plan}

\begin{quotation}
``A specific plan must be provided to disseminate the finished training modules \textbf{nationally} and make them \textbf{freely accessible}. In addition, links to these modules will be posted and maintained on the NIGMS web site."
\end{quotation}

We will create an online tutorial book, since providing tutorials, example code, and example datasets can substantially improve the ability of new users to learn software tools \cite{list2017ten}. We will use GitHub's free ``Pages" web publishing framework to publish the book freely online, and we will also submit it to the \textit{bookdown.org} website under a Creative Commons license. Dr. Anderson (PI) has previously created two \textit{bookdown}-based books, \textit{R Programming for Research} and \textit{Mastering Software Development in R}. Both are publicly and freely available online under the Creative Commons license (see LOS, Xie). 

We will publish the video lectures using the YouTube platform and embed these videos within the online book. The videos, like the book, will be published under a Creative Commons license.
    
\subsection{Timeline}

\begin{quotation}
``Provide a timeline for \textbf{module development}, \textbf{piloting and refinement}, \textbf{dissemination}, \textbf{evaluation}, and \textbf{maintenance}.  This timeline must propose making the training publicly available within two years of the award date."
\end{quotation}


\clearpage

\section{Works cited}

\bibliographystyle{unsrtnat}
\bibliography{rep_modules}

\clearpage

\section{Environment}

\begin{quotation}
``Will the scientific and educational environment of the proposed program contribute to its intended goals? Is there a plan to take advantage of this environment to enhance the educational value of the program? Is there tangible evidence of institutional commitment? Where appropriate, is there evidence of collaboration and buy-in among participating programs, departments, and institutions?"
\end{quotation}

\begin{itemize}
\item Computers. Access to needed software. Computer services (IT).
\item Library. Access to many recent books online.
\item Teaching expertise?
\item Research Rigor \& Ethics center / training?
\item Tech Transfer
\item Participating departments (and their commitment)
\end{itemize}

\textbf{Video Studio and Editing Bays.} CSU's Morgan Library has a video studio and editing bays that can be used by anyone affiliated with CSU to record and edit video content (\url{https://lib.colostate.edu/technology/video-studio-editing-bays/}). These can be reserved and used for free for up to three hours at a time. This facility includes microphones, video lights, video recording equipment, and video editing software.

\textbf{Computer Assisted Teaching Support (CATS) Laboratory.} Staff and facilities for professional video recording and editing. \textit{If we need a cost share for this proposal, their contribution could potentially be in-kind.}

\end{document}
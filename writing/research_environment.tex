\documentclass[pdftex,english,11.5pt,parskip=half]{scrartcl}
\usepackage{palatino}
\usepackage{mathpazo}
\usepackage[margin=0.55in]{geometry}
%\usepackage{parskip}
\usepackage[compact]{titlesec}
\usepackage{amsmath,amssymb}
\usepackage{graphicx}
\usepackage{babel}
\usepackage{framed}
\usepackage{wrapfig}
\usepackage{subfig}
\usepackage[labelfont=bf,font=small,format=plain]{caption}
\usepackage{doi}
\usepackage{booktabs}
\usepackage{longtable}
\usepackage{multirow}
\usepackage[table]{xcolor}
\usepackage{wrapfig}
\usepackage{colortbl}
\usepackage{pdflscape}
\usepackage{tabu}
\usepackage{threeparttable}
\usepackage{threeparttablex}
\usepackage{array}
\usepackage[normalem]{ulem}
\usepackage{makecell}
\usepackage{float}
%\usepackage[authoryear]{natbib}
\usepackage[numbers]{natbib}
\usepackage{url,hyperref,color}
\definecolor{darkblue}{rgb}{0.0,0.0,0.75}
\hypersetup{colorlinks,breaklinks,
            linkcolor=darkblue,urlcolor=darkblue,
            anchorcolor=darkblue,citecolor=darkblue}
\newcommand{\fixme}[1]{{\color{red} #1}}
\renewcommand\thesection{\Alph{section}}
\renewcommand{\familydefault}{\sfdefault}
\newcommand{\tabitem}{~~\llap{\textbullet}~~}
\begin{document}
\addtokomafont{section}{\large}
\def\bf{\normalfont\bfseries}
\pagestyle{empty}

{\large \textbf{Facilities \& Other Resources}}

% \begin{quotation}
% ``Will the scientific and educational environment of the proposed program contribute to its intended goals? Is there a plan to take advantage of this environment to enhance the educational value of the program? Is there tangible evidence of institutional commitment? Where appropriate, is there evidence of collaboration and buy-in among participating programs, departments, and institutions?"
% \end{quotation}
% 
% \begin{quotation}
% ``Describe the educational environment, including the facilities, laboratories, participating departments, computer services, and any other resources to be used in the development and implementation of the proposed program. List all thematically-related sources of support for research training and education following the format for Current and Pending Support."   
% \end{quotation}

Colorado State University is a Research I institution, with vibrant research programs in a variety of scientific, engineering, and health-related fields. As Colorado’s land-grant university, Colorado State University has a 100-year-old extension program to help researchers disseminate their research results to members of the wider community. 

\textbf{Participating Departments.} The Principal Investigator and all Co-Investigators are faculty in Colorado State University's College of Veterinary \& Biomedical Sciences, in either the Department of Environmental \& Radiological Health Sciences or the Department of Microbiology, Immunology, \& Pathology. We will recruit participants for on-campus pilot testing sessions from these two departments, particularly from the Department of Microbiology, Immunology, \& Pathology, which includes over 90 faculty members, and their related research groups, most of whom are involved in biomedical research. Chairs of both departments have expressed their support for the proposed project, as has Colorado State University's Vice President for Research (see letters from Drs. Nickoloff, Dean, and Rudolph).

\textbf{Colorado State University Science, Technology, Education, and Mathematics (STEM) Center.} Colorado State University's STEM Center assists STEM with education-based program and research, including designing and conducting evaluations of these programs. One of the senior evaluators at this center, Dr. Julie Maertens, has been budgeted on this grant to assist with program evaluation.

\textbf{Computer Assisted Teaching Support (CATS) Laboratory.} This laboratory is part of Colorado State University's Academy for Teaching and Learning. The facility includes staff and equipment for professional video recording and editing, which we will use to record the video and audio components of the proposed training materials (see letter from Dr. Andrew West). 

\textbf{Lory Student Center.} Colorado State University's Lory Student Center is in the heart of the Colorado State University campus. It includes meeting rooms that can be freely reserved for the day-long on-campus piloting sessions that we propose to conduct to help evaluate and refine the training modules (see letter from Dr. Jac Nickoloff).

\textbf{Morgan Library.} All key Colorado State University personnel will have full access to Colorado State University’s Morgan Library system. In addition to print holdings, this library also provides 24-hour access to many eBooks and electronic databases of academic articles, including Web of Science and LexisNexis Academic. The library provides additional access to items outside its collections through the Prospector system, which allows Colorado State University faculty to borrow materials from numerous participating academic and public libraries in Colorado and Wyoming. 

\textbf{Computer Services.} The computer hardware and software resources that
will be required to meet the goals of this proposal are already in place and
available to each of the key personnel.  The PI and each Co-PI have recent Colorado State University
/ CVMBS department provided Apple MacBook or Hewlett Packard laptop computers
using either macOS or Microsoft Windows operating systems, and all with the
open
source R programming language and Rstudio integrated development environment
(IDE) installed.  Additionally one of the Co-I's (Dr. Lyons) maintains a
computer lab in the Colorado State University Microbiology Building that has available for this
project, if needed, two Colorado State University network-connected Hewlett Packard HP ProDesk
600 G3 servers equipped with Intel i5 6500 3.2 GHz processors, 1 TB of
storage and 16GB Ram,  both running the linux operating system, both with
git version control, the R programming language, and additional
general purpose scripting and programming languages (Python + NumPy/SciPy),
C and C++ compilers and libraries (gcc, BLAS), the relational database
management system MySQL, the GNU MCSim suite of simulation software, the
gnuplot graphical software, and the LaTeX document preparation system.

Other computing resources (all accessible via network connections for each
of the key personnel) are available with Colorado State University core facilities, with
High-Performance Computing (HPC) infrastructure to support research and
related activities. This includes the CVMBS Research IT Group, with Dell
PowerEdge R720 dual 12 core Xeon processors, 256 GB RAM, 13.5 TB disk space
fast OS (15K RPM) and data (10K RPM) disks.  Linux OS; Four HP Z420
workstations 8-core Xeon
processor, 32 GB RAM, 6 TB disk space Windows 7 OS native, and Linux as a
Virtual Machine.  Central IT at Colorado State University operates, maintains and supports a 2,560 core Cray XE6, with 32 GB of RAM per node, utilizing the high-speed Gemini interconnect. Scratch storage consists of 32 TB of fast, parallel lustre storage, augmented by approximately 100 TB of expandable user storage space. The system is connected at 10 Gbps to the ultrahigh-speed research DMZ LAN. All key personnel will also have access to Colorado State University’s on-site computing support (Colorado State University Computer Diagnostic Center, Colorado State University I.T. Technical Helpdesk). As faculty of Colorado State University, the key personnel will have access to free or reduced-price site-licensed software, including Symantec AntiVirus, Adobe, EndNote, and Microsoft Office.


\textbf{Facilities} All key personnel on this grant from Colorado State University will have office space within the Department of Radiological \& Environmental Health Sciences of Colorado State University, the Department of Microbiology, Immunology, \& Radiology, or Colorado State University's STEM Center throughout the course of the proposed research. At these offices, all key personnel will have access to department staff to aid in submitting and reporting on grants, as well as access to unlimited free printing, copying, and scanning.

\end{document}
